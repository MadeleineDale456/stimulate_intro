\documentclass[11pt,t,xcolor=dvipsnames,aspectratio=169]{beamer}

\usetheme{Custom}

\usepackage{sansmathfonts}

%\usepackage[italian]{babel}
\usepackage[T1]{fontenc}
\usepackage[utf8]{inputenc}

\usepackage{mathtools}

\usepackage{tikz}
\usetikzlibrary{decorations,decorations.markings}
\usetikzlibrary{decorations.shapes}
\usepackage{import} % inputs with relative path
\let\pgfimageWithoutPath\pgfimage 
\renewcommand{\pgfimage}[2][]{\pgfimageWithoutPath[#1]{gfx/#2}}


\title[Fellow project presentation]{Hybrid quantum-classical simulations of lattice gauge theories}
\author{Giovanni Iannelli}
%\supervisor{}
\institute{Humboldt University of Berlin}
\date{15 January 2019}

\newlength\leftsidebar
\makeatletter
\setlength\leftsidebar{\beamer@leftsidebar}
\makeatother

\begin{document}

\hoffset=-0.5\leftsidebar
{
    %\usebackgroundtemplate{
    %    \put(200,-200){%
    %    \tikz\node[opacity=0.2] {\pgfuseimage{humboldt}};
    %    }

    %    \put(30,-220){%
    %    \tikz\node[opacity=0.2] {\pgfuseimage{cyprus}};
    %    }

    %    \put(350,-230){%
    %    \tikz\node[opacity=0.2] {\pgfuseimage{torvergata}};
    %    }
    %}
\begin{frame}[plain,t]
\titlepage
\end{frame}
}
\hoffset=0in % restore left margin

%\section*{}
%\begin{frame}[c]
%  \frametitle{Sommario}
%  \tableofcontents[hidesubsections]
%\end{frame}


\tikzset{->-/.style={decoration={markings,mark=at position .55 with {\arrow{to}}},postaction={decorate}}} 
\tikzset{
    partial ellipse/.style args={#1:#2:#3}{
        insert path={+ (#1:#3) arc (#1:#2:#3)}
    }
}

\section{Background}

\begin{frame}
    \frametitle{Some Background}
    \begin{itemize}
        \item
            I completed my Bachelor's and Master's in Physics at Pisa University.
        \item
            I focused on Quantum Mechanics, Quantum Field Theories and Statistical Physics, in particular on their computational aspects.
        \item
            During my Master's, I joined the Guest Student Programme in J\"ulich Supercomputing Centre.
        \item
            Apart from Theoretical Physics and HPC, my main other interests are Probability Theory, Statistics and Machine Learning.
    \end{itemize}
\end{frame}

\section{Experiences}

\begin{frame}
    \frametitle{Previous experiences}
    \begin{itemize}
        \item
            For my Bachelor thesis, I implemented a MC simulation of a Bose-Einstein condensate in a harmonic trap with Prof. E. Vicari.
        \item
            During my stay in J\"ulich, I implemented the Schwinger model on a Klein bottle, and studied the properties of the topological charge with Prof. K. Szabo.
        \item
            For my Master thesis, 
            I developed and evaluated with Prof. M. D'Elia
            a new cluster algorithm for the Schwinger model that solves the problem of topological freezing in this system.
        \item
            Further studies are being done to extend this algorithm to more advanced models.
    \end{itemize}
\end{frame}

\section{Project}

\begin{frame}
    \frametitle{Quantum simulation with quantum computers}
    \begin{itemize}
        \item
            We working with Rigetti and IBM Q quantum computers.
        \item
            It is possible to map some physical models to qubits systems.
        \item
            We will start with 1+1 dimensional models that are of interest in condensed matter and high energy physics:
            Heisenberg model, Schwinger model, $\mathbb CP^{N-1}$ model and Gross-Neveu model.
        \item
            Instead of evaluating path integrals with MC simulations,
            we want to perform direct measures on qubits, avoiding problems such long autocorrelation and the sign problem.
        \item
            We plan to apply the same idea to 2+1 dimensional gauge theories, abelian and non-abelian, and coupled to matter.
    \end{itemize}
\end{frame}

\begin{frame}
    \frametitle{Quantum advantage}
    \begin{itemize}
        \item
            There is a Quantum advantage if a quantum computer can solve a problem that a classical computer cannot.
        \item
            Not achieved yet. IBM Q, Rigetti, Google and NASA are working on it for years.
        \item
            We expect that quantum physics problems will have greater advantages from quantum computers:
            \begin{itemize}
                \item Quantum systems are easier to map to qubits states. They share QM rules.
                \item Integrating path integrals is very resource demanding.
            \end{itemize}
        \item
            System with the sign problem could show quantum advantages:
            \begin{itemize}
                \item Path integrals evaluation becomes very complicated. In some cases impossible.
                \item Even an imprecise result with a quantum computer could show a quantum advantage.
            \end{itemize}
    \end{itemize}
\end{frame}

\hoffset=-0.5\leftsidebar
\begin{frame}[plain,t]
\titlepage
\end{frame}
\hoffset=0in % restore left margin

\section{Appendix}

\begin{frame}
    \frametitle{The sign problem in computational physics}
    \begin{itemize}
        \item
            The sign problem arises when evaluating highly oscillatory path integrals.
        \item
            It's one of the main problems in computational physics.
            It affects MC simulations of:
            \begin{itemize}
                \item Many electrons systems at low temperatures.
                \item Nuclei and neutron stars.
                \item Quark matter and vacuum in QCD.
            \end{itemize}
        \item
            Quantum simulators don't suffer from it.
            They don't need numerical integration of path integrals.
            Measures are performed directly on quantum states.
        \item
            Quantum computers may provide a complementary numerical method in these fields.
    \end{itemize}
\end{frame}

\end{document}
